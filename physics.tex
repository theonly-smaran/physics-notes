\documentclass[openany]{book}
\usepackage{textcomp}
\usepackage{gensymb}
\usepackage{hyperref}
\usepackage{mathtools}
\usepackage{amsmath}
\usepackage{bm}
\usepackage{esvect}
\newcommand{\ihat}{\hat{\textbf{\i}}}
\newcommand{\jhat}{\hat{\textbf{\j}}}
\newcommand{\khat}{\hat{\textbf{k}}}
\title{
\centering
\rule{\textwidth}{2pt}
\break
{\bfseries\Huge \centering Physics Notes}
\rule{\textwidth}{2pt}
}
\author{
Smaran Timsina
}
\date{September 21, 2025 - Present}

\begin{document}
\maketitle
\tableofcontents
\chapter{Preface \& Prerequisites}
\quad Hi there! I'm making this resource for anyone trying to learn physics whether it be for a class or just for funsies!! This resource is primarily for AP Physics (1/C) level students, as well as those studying for any of the numerous Physics Olympiads. It's also a documentation of my own personal journey. I hope that for whatever purposes you may use this that you are successful and lock in! 


For these notes, I'll be using a textbook called {\bfseries\textit{Physics}} by David Halliday, Robert Resnick, Jearl Walker (commonly known as {\bfseries\textit{HRK}} in the Physics Olympiad community). If you'd like to follow along with it or use it on your own, you can find it available to buy through different online stores, like Amazon and Thriftbooks.


To understand much of this course, you should be familiar with concepts like trigonometry (right triangle, unit circle, trigonometric functions, trig IDs, laws of sines and cosines, etc.), algebra (graphing functions, solving equations, finding roots of polynomials, complex roots and powers, logarithmic and exponential functions), and optionally, I recommend you have already learned or are learning calculus. I also recommend you have great attention to detail and a good sense of direction!


\chapter{Kinematics}
\section{Introduction}
\quad This chapter will be about kinematics. Kinematics is the study of motion in physics, as we abstract objects in order to better and more easily understand their motion. In this chapter, I'd recommend having a good understanding of derivatives (rate of change/slope) as well as vector math (literally just some trig and Pythagorean theorem). 

\section{Kinematics with Vectors}

\begin{itemize}
\item {\bfseries{Vectors}} have both magnitude {\bfseries{and}} \textit{direction}.
\item {\bfseries{Scalars}} only have magnitude.
\item {\bfseries{Kinematics}} is the study of {\bfseries{motion}}: position, velocity, and acceleration, in particular.
\item In this chapter, objects will be treated as {\bfseries{particles}} or {\bfseries{point-objects}}, to ignore internal or rotating motion to just focus on motion along axes.
\end{itemize}

\section{Properties of Vectors}
\begin{itemize}
    \item Graphically, {\bfseries{Arrow}} = {\bfseries{Vector}}
    \item {\bfseries{Length}} of vector arrow is {\textit{proportional}} to its magnitude, use some scale to define the proportionality (ex. some length on paper is equivalent to this length in reality)
    \item Vectors can be represented as: $\vec{a}$
    \item The magnitude of a vector can be represented as: $|\vec{a}|$ or an italicized character, {\textit{a}}
    \item Now to use vectors to a further capacity, we'll have to use vector components to see how an object, especially one moving along more than one axis moves in a specific component to analyze its motion. For example, in a projectile, its important to analyze horizontal and vertical components separately to understand its motion better as they are independent of one another.
    \item In 2D Kinematics, we can define the components of a vector using the vector's magnitude, $|\vec{a}|$, and its direction, defined by an angle, $\theta$, relative to some reference point.
    \item You can define these components mathematically as the following: \break $\vec{a_x}$ = $|\vec{a}|\cos{\theta}$ and $\vec{a_y}$ = $|\vec{a}|\sin{\theta}$
    \item If you measure $\theta$ to be an angle in the domain [0$^{\circ}$, 90$^{\circ}$] then change the sign of $|\vec{a}|$ to get the correct values of your components, as the sign matters (since its a vector and vectors need a direction, sign = direction), or just use the actual angle as the trigonometric multiplier will cause it to negate if needed.
    \item If you have other parts of a vector you can get its other parts too. If you have the two components but not the magnitude or direction, you can find magnitude by applying the Pythagorean theorem:
    \\
    $|\vec{a}|$ = $\sqrt{\vec{a^{2}_x} + \vec{a^{2}_y}}$
    \\
    You can find the direction by applying the inverse tangent function:
    \\
    $\tan{\theta} = \frac{\vec{a_y}}{\vec{a_x}} 
    \Longrightarrow \theta = \tan^{-1}(\frac{\vec{a_y}}{\vec{a_x}})$
    \item You can also write vectors in terms of unit vectors, indicated by ${\ihat}$ and ${\jhat}$. So instead of writing your vectors in the common notation of defining the magnitude and associating it with the direction its pointing in, you can just write:
    \\
    $\vec{a} = \vec{a_x}{\ihat} + \vec{a_y}{\jhat}$
    \item We can now refer to $\vec{a_x}{\ihat}$ and $\vec{a_y}{\jhat}$ as our vector components
    \item To add vectors, add the matching components of each, so put the x-components with the x-components and the y-components with the y-components.
    \item If you multiply a vector by a scalar, it will stretch out or shrink the vector, but it will remain in the same direction unless the scalar is negative, then the direction will be opposite and flipped.
\end{itemize}
\section{Position, Velocity, and Acceleration Vectors}
\begin{itemize}
    \item Using an xyz coordinate system, we can consider particles moving in three dimension and define the displacement of the object, $\vec{r} = x{\ihat} + y{\jhat} + z{\khat}$
    \item To find a displacement vector we just subtract one vector from another, which is the same thing as adding opposite (negative) of the second vector to the first one:
    \\
    $\Delta\vec{r} = \vec{r_2} - \vec{r_1}$ or
    $\Delta\vec{r} = \vec{r_f} -\vec{r_0}$
    \item Average velocity is defined by the displacement over the change in time, or the change in position over the change in time:
    \\
    $\vec{v_{avg}} = \frac{\Delta\vec{r}}{\Delta{t}}$, where $\Delta{t} = t_f - t_0$
    \item Even if you travel far away if you end up where you started your position technically hasn't changed from start to finish. That's why on circular tracks, if you run a lap, your displacement {\bfseries{and}} average velocity are both equal to 0.
    \item Instantaneous velocity is the velocity at an instant. This is represented as the derivative of the position with respect to time. We can define it with the limit definition of the derivative:
     
    \[\vec{v} = \lim_{t\to0} \frac{\Delta\vec{r}}{\Delta t} \]

    We can also define it with the standard definition of the derivative:

    $\vec{v} = \frac{d\vec{r}}{dt}$

    \item And since we've defined the value of $\vec{r}$ we can substitute it into the derivative of the position in respect to time by doing the following:
    
    $\vec{v} = \frac{d\vec{r}}{dt} = \frac{d}{dt}(x{\ihat} + y{\jhat} + z{\khat})$

    We can now just take the derivative of each term to get

    $\vec{v} = \frac{dx}{dt}{\ihat} + \frac{dy}{dt}{\jhat} + \frac{dz}{dt}{\khat}$

    \item Additionally, since velocity is also a vector, we can write it in the unit vector notation of a three-dimensional (particle) object's instantaneous velocity to get:
    
    $\vec{v} = {\vec{v_x}}{\ihat} + {\vec{v_y}}{\jhat} + {\vec{v_z}}{\khat}$

    If you pay attention you may notice a similarity here between each component of the vector sum of the instantaneous velocity and that of the components of the derivative of the components of the point-object's position. You'll notice that the derivative of each component of the displacement is equal to the matching component of the particle's instantaneous velocity. So, 

    $\vec{v_x} = \frac{dx}{dt}, \qquad \vec{v_y} = \frac{dy}{dt}, \qquad \vec{v_z} = \frac{dz}{dt}$

    \item Instantaneous speed is the magnitude of the instantaneous velocity, $|\vec{v}|$
    \item Average speed is the total distance traveled over the time elapsed:
    
    average speed = $\frac{\text{distance traveled}}{\Delta t}$

    \item Acceleration is the change in velocity over some time, this can mean a number of things, but in general, most of the time, we'll see acceleration in a scenario where an object is slowing down ($\vec{v_f} < \vec{v_0}$), speeding up ($\vec{v_f} > \vec{v_0}$), or changing direction (in 1DK $\Rightarrow$ primarily a change in sign, 2-3DK $\Rightarrow$ change is more ambigous, can usually tell by change in reference angle)
    \item Average Acceleration can be written as change in velocity over change in time:
    
    $\vec{a_{avg}} = \frac{\Delta \vec{v}}{\Delta t}$

    \item Instantaneous acceleration can also be written using derivatives like we did with instantaneous velocity:
    
    \[\vec{a} = \lim_{t\to0} \frac{\Delta\vec{v}}{\Delta\vec{t}} \]


    \[\vec{a} = \frac{d\vec{v}}{dt} \]

    And using the same logic from earlier, the instantaneous acceleration can be broken into vecotr components, $\vec{a_x}, \vec{a_y}, $ and $\vec{a_z}$. Which can each be defined as the derivative of the instantaneous velocity of the matching component. Therefore,

    \[\vec{a_x} = \frac{d\vec{v_x}}{dt}, \qquad \vec{a_y} = \frac{d\vec{v_y}}{dt}, \qquad \vec{a_z} = \frac{d\vec{v_z}}{dt} \]

\end{itemize}
    
    \section{1DK, Accelerated (Constant) Motion, Freefall}
    \quad Lowk I'm getting too lazy to write everything because this is pretty rudimentary stuff, just do HRK's exercises to practice your skills. Here are the formulas you have to know for the rest of the unit:

    \begin{itemize}
        \item $\vec{v_f} = \vec{v_0} + \vec{a}t$
        \item $\Delta x = \vec{v_0}t + \frac{1}{2}\vec{a}t^{2}$
        \item $\vec{v_f^{2}} = \vec{v_0^{2}} + 2\vec{a}\Delta x$
        \item For each vector quantity, remember to make sure that you're using all of them in the same component, as each component is independent of each other. In simpler words, let's say, that you're solving for parts in the vertical motion of an equation, if so, make sure all of your vectors are in the y-component form.
    \end{itemize}

\chapter{Force and Newton's Laws}
\section{Classical Mechanics}
    \begin{itemize}
    \item The study of classical mechanics is just how objects interact in its environment, within a system, so that the object's velocity will increase (an acceleration is produced)
    \item We will continue to treat objects as point-objects or particles so for now, we will ignore internal structure and motion of an object and assume all parts of it work the same way
    \item Most classical mech. problem consist and work the same way:
    \begin{itemize}
        \item An object that has some known physical properties like mass, volume, electric charge, etc.
        \item Said object is placed at some known initial location, mlving along with a known initial velocity, $\vec{v_0}$
        \item We know or are then able to measure all of the interactions of the object with the environment
    \end{itemize}
    \item We use these analysis technqiues to find the subsequent motion (future position and velocity).
    \item Interaction of an object with its environment (or other objects within it) is called a force, denoted by $\vec{F}$, a force is essentially a push or a pull on an object by another object, to put it simoply. Force is a vector so understand that it will need direction specificied as well. Additionally, to add forces, you need to use vector addition (basic trig).
    \item Again every force on some object is caused by a particular body/object in its environment.
    \item Specify each force as what it acts on as well as what is causing the force. For example:
    \begin{itemize}
        \item pushing force \textit{on} create \textit{by} worker
        \item frictional force \textit{on} create \textit{by} floor
        \item gravitational (or weight) force \textit{on} crate \textit{by} the Earth
    \end{itemize}
    \item To even start solving classical mechanics we have to begin by defining the magnitude of a force in terms of the acceleration of a particular standard body upon which that force acts. We then assign a mass, m, to a body by comparing the acceleration of that body with the acceleration of the standard body.
    \item To end this section, understand that force is an interaction {\bfseries{between}} a body (object) {\bfseries{and}} its environment. The study of accelerated motion of an object, is {\bfseries{caused}} by a force {\bfseries{acting}} upon said object.
    \end{itemize}
\section{Newton's First Law} 
\subsection{Concept of Inertia}
\begin{itemize}
    \item Since a force acting upon an object causes accelerated motion, it's reasonable that we can claim that if no force is acted, no accelerated motion will happen.
    \item This can be two scenarios, one, where an object doesn't move at all, as no displacement means no velocity, if velocity never changes and remains zero, no acceleration will happen. The other scenario is if we're already moving at a constant velocity, where if there's no force acting on said object with constant velocity, then the object will never stop nor speed up, and will continue moving with the same velocity.
    \item It's very hard emulate the second scenario in real life as there are usually at least a couple forces acting on an object if not many, like the weight or gravitational force, acted on an object by the Earth. However, it's possible to emulate the first scenario, using the concept of net force, $\vec{F_{net}}$
    \item Net force, $\vec{F_{net}}$ can be defined as the vector sum of all forces acting upon an object, so if we have each force being canceled out with a force with equal magnitude and with opposite direction, we'll achieve a scenario where there is no force as each force counteracts the other.
    \item This is actually pretty common in real life, as it models objects at rest.
    \item Ultimately, this concept is summarized through Newton's first law of motion: {\textit{Consider a body on which no net force acts. If the body is at rest, it will remain at rest. If the body is moving with constant velocity, it will continue to do so.}}
\end{itemize}
\subsection{The First Law and Reference Frames}
\begin{itemize}
    \item You want to be in a inertial reference frames when determining forces
    \item To be in an inertial reference frame means that you're either at rest or at constant velocity and continue to stay that way
    \item The best way to know why you want to be in an inertial frame of reference is to think of the motion of a car. So let's say you're driving a car at a constant speed and you slam the brakes, your car skidding to a stop. To you, it may look like an object on the seat next to you, like a book, is sliding towards the front of the car because you've slowed down very quickly. If you were to analyze the motion from your point, this wouldn't make much sense, as a force that would have to push the book would be either some sort of contact force which doesn't happen, or a long-distance force, which when studying mechanics is primarily the force of gravity. Now in this scenario, {\bfseries{neither}} is true, so what force would be acting on it? None. There's no actual force acting on the book itself causing it to accelerate forward. There is a force of kinetic friction acting on the car, causing it to slow the car down, {\bfseries{however}}, that's an interaction between the car and the road, not the book and the surface it's on in the car. In this case, there would be some {\bfseries{\textit{fictitious}}} force acting upon the book, which isn't real. What's actually happen, if you think about it long enough, is that your book is going the same speed as the car but it doesn't have the kinetic friction against the road slowing it down, so, due to Newton's First Law, it will continue that same motion, causing it to seem like a force is pushing it forward when it's just trying to continue its motion, well at least until it's stopped by something physically.
\end{itemize}
\subsection{Inertial Reference Frames \&\ Relative Motion}
\begin{itemize}
    \item Okay legit the big concept here is that, as long as we considering observers are in inertial frames of reference, depending on where you are, you'll observe different velocities and positions
    \item For example, let's say you're in a car moving at a constant 50 mph and your friend is on the side of the road. If a car passes by, moving with a constant velocity that's actually 70 mph (relative to the ground), you'll notice that from you're perspective its moving 20 mph, because thats how it's velocity compares to yours. Your friend will view it as 70 mph because he can see easily that it's moving that speed as your friend is at the side of the road.
    \item An even easier way to visualize this is if you notice on a road, if you and another car are going the same speed, it'll look as if the other car isn't moving at all (if we take away the environment around you like everything that makes you notice you're moving).
    \item Using kinematics, you can calculate acceleration too using relative velocities
    \item The acceleration from any inertial observer of the same scenario should be the same (although velocities and positions won't be the same).
    \item In the first law, there isn't much of a difference between an object at rest and one with constant velocity (a net force of {\bfseries{zero}}/no net force is acted upon the object).
\end{itemize}
\subsection{Force}
\quad Overall, this chapter is pretty simple, there are like just a couple main ideas you should have a good grasp on.
\begin{itemize}
    \item Intrinsically, a force is defined as the acceleration it'll produce on a standard body. A standard body is an object with mass of 1 kg. For example, it'll take one unit of force (1 N) to produce an acceleration of 1 m/s$^{2}$, and it'll take two units of force (2 N) to produce an acceleration of 2 m/s$^{2}$ on the object. This definition itself should be rather intuitive, nothing you gotta memorize. This definition also helps you understand force better, as most people try to understand it by the formula given in Newton's Second Law, \[\Sigma\vec{F_i} = m\vec{a_i}\] Now, this formula is mathematically correct and we use it more often than the definition that's defined in order to analyze forces, however, to understand what a force is and does itself, it's important to look at it from the mathematical definition we get from manipulating this equation (not required but definently helps to understand better): \[\vec{a_i} = \frac{\Sigma\vec{F_i}}{m}\]
    \item ({\textit{Note:}} the subscript "i" just substitutes the component axis that you're looking at, so if you're looking at the x components for a particular force and motion, substitute in an x)
    \item The sum of forces, $\Sigma\vec{F}$, represents the {\bfseries{vector}} sum of all forces acting upon an object
    \item We can use this information to gather more information on some problems intuitively. Let's say that we know that some object is moving at a constant speed, if so, we know that all of its forces are balanced. This is because no acceleration is produced, and we can verify this using Newton's First Law, as from it we can safely say that unless some force is introduced, the motion will remain the same, and we aren't introducing any new force.
    \item Lastly, although this concept isn't discussed in HRK, I think it's important enough to mention. Unless a force is a long-distance force (gravitational and electromagnetic, at least to the best of my knowledge heh) it has to be in contact to be counted as a force when analyzing. This is pretty simple to understand intuitively as these forces are literally called contact forces but a lot of people still forget that, and let me repeat again, when analyzing these forces on an object, they {\bfseries{NEED}} to be {\bfseries{IN CONTACT}} with said object.
\end{itemize}
\end{document}